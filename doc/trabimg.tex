%% Author: Roberto Gerson de Albuquerque Azevedo
%% Chapter Processamento de Imagens

\chapter{Processamento de Imagem}

\section{Operações Geométricas}
\par
O propósito das operações geométricas é deformar uma imagem alterando sua
geometria, diferente dos filtros, onde a posição dos pixels não se altera,
apenas sua intensidade. Exemplos comuns de operações geométricas são: rotação,
escala e translação. Em especial, as operações geométricas são de suma
importância na computação gráfica, onde comumente texturas devem ser deformadas
para mapear em uma superfície 3D.

\par
Diferente do que inicialmente possa parecer as operações geométricas não são
simples de serem realizadas. Em um exemplo bem trivial como em escalar a imagem
já é possível perceber que se quisermos escalá-la de um fator $n$, não teremos
um resultado razoável se replicarmos cada pixal $n$ x $n$ vezes.

\begin{center}
\begin{tabular}{ l c r }
  Classe & quadrado & círculo \\
  Geral & Arbitário & Arbitrário \\
  Projetiva & Quadrilátero & Curva cônica \\
  Afim & Paralelograma & Elipse \\
\end{tabular}
\end{center}

\subsection{Perspective}

\subsection{Twirl}

\subsection{Sphere}

\section{Interpolação}
\subsection{Linear}
\subsection{Bilinear}

\section{Adicional: Mapeamento de Texturas}
\subsection{Filtro EWA}
\par
foreach pixels in screen
{
  calculate jacobian matrix (a local affine transform to the)
}

